%%%%%%%%%%%%%%%%%%%%%%%%%%%%%%%%%%%%%%%%%
% Wilson Resume/CV
% XeLaTeX Template
% Version 1.0 (22/1/2015)
%
% This template has been downloaded from:
% http://www.LaTeXTemplates.com
%
% Original author:
% Howard Wilson (https://github.com/watsonbox/cv_template_2004) with
% extensive modifications by Vel (vel@latextemplates.com)
%
% License:
% CC BY-NC-SA 3.0 (http://creativecommons.org/licenses/by-nc-sa/3.0/)
%
%%%%%%%%%%%%%%%%%%%%%%%%%%%%%%%%%%%%%%%%%

%----------------------------------------------------------------------------------------
%	PACKAGES AND OTHER DOCUMENT CONFIGURATIONS
%----------------------------------------------------------------------------------------

\documentclass[10pt]{article} % Default font size
\usepackage[colorlinks = true,
            linkcolor = gray,
            urlcolor  = gray,
            citecolor = gray,
            anchorcolor = gray]{hyperref}
 \usepackage[usenames,dvipsnames]{xcolor}



%%%%%%%%%%%%%%%%%%%%%%%%%%%%%%%%%%%%%%%%%
% Wilson Resume/CV
% Structure Specification File
% Version 1.0 (22/1/2015)
%
% This file has been downloaded from:
% http://www.LaTeXTemplates.com
%
% License:
% CC BY-NC-SA 3.0 (http://creativecommons.org/licenses/by-nc-sa/3.0/)
%
%%%%%%%%%%%%%%%%%%%%%%%%%%%%%%%%%%%%%%%%%

%----------------------------------------------------------------------------------------
%	PACKAGES AND OTHER DOCUMENT CONFIGURATIONS
%----------------------------------------------------------------------------------------

\usepackage[a4paper, hmargin=10mm, vmargin=15mm, top=10mm]{geometry} % Use A4 paper and set margins

\usepackage{fancyhdr} % Customize the header and footer

\usepackage{lastpage} % Required for calculating the number of pages in the document

\usepackage{hyperref} % Colors for links, text and headings

\setcounter{secnumdepth}{0} % Suppress section numbering
\usepackage{enumerate}

%\usepackage[proportional,scaled=1.064]{erewhon} % Use the Erewhon font
%\usepackage[erewhon,vvarbb,bigdelims]{newtxmath} % Use the Erewhon font
\usepackage[utf8]{inputenc} % Required for inputting international characters
\usepackage[T1]{fontenc} % Output font encoding for international characters
\usepackage{nopageno}
\usepackage{fontspec} % Required for specification of custom fonts
\setmainfont[Path = ./fonts/,
Extension = .otf,
BoldFont = Erewhon-Bold,
ItalicFont = Erewhon-Italic,
BoldItalicFont = Erewhon-BoldItalic,
SmallCapsFeatures = {Letters = SmallCaps}
]{Erewhon-Regular}

\usepackage{color} % Required for custom colors
\definecolor{slateblue}{rgb}{0.17,0.22,0.34}

\usepackage{sectsty} % Allows customization of titles
\sectionfont{\color{slateblue}} % Color section titles

\usepackage{setspace}
\usepackage{titlesec}

\fancypagestyle{plain}{\fancyhf{}\cfoot{\thepage\ of \pageref{LastPage}}} % Define a custom page style
\pagestyle{plain} % Use the custom page style through the document
\renewcommand{\headrulewidth}{0pt} % Disable the default header rule
\renewcommand{\footrulewidth}{0pt} % Disable the default footer rule

\setlength\parindent{0pt} % Stop paragraph indentation

% Non-indenting itemize
\newenvironment{itemize-noindent}
{\setlength{\leftmargini}{0em}\begin{itemize}}
{\end{itemize}}

% Text width for tabbing environments
\newlength{\smallertextwidth}
\setlength{\smallertextwidth}{\textwidth}
\addtolength{\smallertextwidth}{-2cm}

\newcommand{\sqbullet}{~\vrule height 1ex width .8ex depth -.2ex} % Custom square bullet point definition

%----------------------------------------------------------------------------------------
%	MAIN HEADER COMMAND
%----------------------------------------------------------------------------------------

\renewcommand{\title}[1]{
{\huge{\color{slateblue}\textbf{#1}}}\\ % Header section name and color
\rule{\textwidth}{0.5mm}\\ % Rule under the header
}

%----------------------------------------------------------------------------------------
%	JOB COMMAND
%----------------------------------------------------------------------------------------

\newcommand{\job}[6]{
\begin{tabbing}
\hspace{2cm} \= \kill
\textbf{#1} \> \href{#4}{#3} \\
\textbf{#2} \>\+ \textit{#5} \\
\begin{minipage}{\smallertextwidth}
\vspace{0mm}
#6
\end{minipage}
\end{tabbing}
\vspace{0mm}
}

%----------------------------------------------------------------------------------------
%	SKILL GROUP COMMAND
%----------------------------------------------------------------------------------------

\newcommand{\skillgroup}[2]{
\begin{tabbing}
\hspace{5mm} \= \kill
\sqbullet \>\+ \textbf{#1} \\
\begin{minipage}{\smallertextwidth}
\vspace{0mm}
#2
\end{minipage}
\end{tabbing}
}

%----------------------------------------------------------------------------------------
%	INTERESTS GROUP COMMAND
%-----------------------------------------------------------------------------------------

\newcommand{\interestsgroup}[1]{
\begin{tabbing}
\hspace{5mm} \= \kill
#1
\end{tabbing}
\vspace{0mm}
}

\newcommand{\interest}[1]{\sqbullet \> \textbf{#1}\\[3pt]} % Define a custom command for individual interests

%----------------------------------------------------------------------------------------
%	TABBED BLOCK COMMAND
%----------------------------------------------------------------------------------------

\newcommand{\tabbedblock}[1]{
\begin{tabbing}
\hspace{2cm} \= \hspace{0cm} \= \kill
\vspace{2mm}
#1
\end{tabbing}
} % Include the file specifying document layout

%----------------------------------------------------------------------------------------
%no page numbering
\pagestyle{empty}
\begin{document}
\titlespacing{\section}{0pc}{1.5ex plus .1ex minus .2ex}{1pc}

%----------------------------------------------------------------------------------------
%	NAME AND CONTACT INFORMATION
%----------------------------------------------------------------------------------------

\title{Taras Murzenkov } % Print the main header

%------------------------------------------------

\parbox{0.2\textwidth}{ % First block
\begin{tabbing} % Enables tabbing
\hspace{3cm} \= \hspace{4cm} \= \kill % Spacing within the block
{\bf Mobile Phone} \>+38 (096)-969-16-49\\ % Mobile phone
{\bf Email} \> \href{mailto:murzenkovtn@gmail.com}{murzenkovtn@gmail.com} \\ % Email address
\end{tabbing}}
\hfill % Horizontal space between the two blocks
\parbox{0.5\textwidth}{ % Second block
\begin{tabbing} % Enables tabbing
\hspace{3cm} \= \hspace{4cm} \= \kill % Spacing within the block
{\bf LinkedIn} \> \href{https://www.linkedin.com/in/tarasmurzenkov}{LinkedIn, click to view} \\ % LinkedIn address
{\bf GitHub} \> \href{https://github.com/tarasmurzenkovv/}{Current projects, click to view}\\ % GitHub address

\end{tabbing}}

{
\textbf{Languages:} Java 8-17, Kotlin, Groovy, SQL/HQL, JavaScript/TypeScript, Dart\newline
\textbf{Frameworks/Libs:} Spring Boot, Quarkus, Hibernate ORM, jUnit, Testcontainers, Mockito, NodeJS/ExperssJS \newline
\textbf{Databases:}  PostgreSQL, ElasticSearch, Cassandra, Redis \newline
\textbf{Build tools:} Gradle, Maven \newline
\textbf{Ops. tools:} Jenkins, Docker, Kubernetes, CircleCI, Bash scripting, Terraform \newline
\textbf{Queues:} Apache Kafka, RabbitMQ \newline
\textbf{Dev. methodologies:} Agile Scrum, Pair programming, TDD \newline
\textbf{Clouds:} AWS, Digital Ocean, Heroku
}
%----------------------------------------------------------------------------------------
%	EMPLOYMENT HISTORY SECTION
%----------------------------------------------------------------------------------------
\section{Recent Positions}
\job
{Oct. 2022 }{ Current}
{Masabi}
{http://masabi.com}
{Senior Software Engineer}
{
\textbf{}   
\\$\bullet$ Designed and implemented RESTfull service that allows remote IoT devices to update their configuration (AWS, Terraform, Kotlin, http4k)
\\$\bullet$ Created the highly scalable event-driven replication service (DynamoDB, Lambda, Terraform, Python)
\\$\bullet$ Created the E2E testing pipeline that allowed to increase the deployment frequency
\\$\bullet$ Incorporated ADR process so that each team member could review the architectural proposal
\\$\bullet$ Introduced API first development by brining the necessary code generators and enhancing the release pipeline
 }
\job
{Nov. 2018 }{ Oct. 2022}
{Intellias}
{http://intellias.ua/}
{Senior Java/Kotlin Developer}
{
\textbf{}   
\\$\bullet$ Took the full design and delivery ownership of the two core features (Customer profile service and Analytics integration service): designed the solution architecture, wrote implementation and rolled out to production. Tech. stack - Java/Kotlin, Spring Boot, Kafka, AWS (EKS, Lambda, DynamoDB, SQS)
\\$\bullet$ Created the highly scalable cross-database replication service (Tech. stack: DynamoDB, Lambda, Terraform, Python)
\\$\bullet$ Scaled CI pipeline by removing manual steps and integrating automatic quality gates and expanding test suite so that the number of deployments increased by 40\% on daily basis 
\\$\bullet$ Reduced by 30\% response latency of the critical service by batching ElasticSearch queries 
\\$\bullet$ Created library to detect N+1 JPQL problem, used it in integration tests so that the overall query latency was improved by 70\% 
\\$\bullet$ Designed and implemented 2 mathematical models as JAR files to asses leaching and run-off risks, 
integrated them in the Sustainability Project so that customers had the ability to apply active ingredients in the sustainable manner
 }
\job
{Oct. 2017}{Nov. 2018}
{EPAM}
{http://epam.ua/}
{Java Developer}
{
\textbf{}    
\\$\bullet$ Reduced the number of http errors by 50 \% in service-to-service communications during unexpected request spikes by integrating reactive libraries 
\\$\bullet$ Improved release process by creating automatic regression E2E test suite, integrating into Jenkins pipeline that lead to 1 man-day reduction in QA team 
\\$\bullet$ Improved the test coverage to 90\% and fixed Sonar Qube issues during Company audit process
\\$\bullet$ Designed, run load testing and optimised importing process by tuning Hibernate sequence generating algorithm so that the throughput was increased by 40\%
\\$\bullet$ Created, integrated service that fetched OFAC data so that the system had the up to date information about sanctioned entities
 }

\job
{Jun. 2015}{Oct. 2017}
{DataArt}
{http://dataart.ua/}
{Java Developer}
{
\textbf{}  
\\$\bullet$ Created RESTfull service that allowed financial analytics to scale model design process 
\\$\bullet$ Refactored Liquibase migration set to allow migration database changes from Oracle to MS SQL Server
\\$\bullet$ Improved unit test coverage to 90\% to support refactoring of the legacy code base
\\$\bullet$ Incorporated code quality metrics to reduce code complexity and improve feature delivery process
\\$\bullet$ Created and executed load testing to access the current system architecture and identify existing bottleneck 
 }


\end{document}
 \usepackage{nopageno}